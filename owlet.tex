\documentclass[11pt]{article}
\usepackage[margin=1in, paperwidth=8.5in, paperheight=11in]{geometry}
\usepackage{graphicx}
\setlength\intextsep{0pt}
\usepackage{enumitem}
\usepackage{tikz}

\setitemize[1]{itemsep=-5pt,topsep=0pt}

\pagestyle{empty}
\setlength{\belowcaptionskip}{-10pt}

\usepackage{graphicx}
\usepackage{hyperref}
\usepackage[
type={CC},
modifier={by},
version={4.0},
]{doclicense}


\pagestyle{empty}
\begin{document}
\begin{centering}
	
\textbf{\LARGE Parliament of Owlets Hat}
\medskip

{\large Veronica Boyce}

\smallskip

\end{centering}


A double knit hat in fingering yarn with 6 owls on the side and a flower shape on the top. Owls use cables, nupp eyes, and color changes. Hat uses increases and decreases for shaping. 
\smallskip

\textbf{Caveats:} This pattern assumes a working familiarity with double knitting and willingness to use hard techniques. This pattern uses the following techniques, which are not explained.
\begin{itemize}
	\item 4 pair x 5 pair cables
	\item nupps
	\item lifted increases
	\item single decreases
\end{itemize}

\smallskip
\textbf{Yarn:} ZZ grams of each of two colors of fingering weight yarn (approx YY yards). Shown in Tenderfoot Superfine (75\% merino lambswool, 25\% polyamide) in colors Linen and Deep Russet. 

\smallskip
\textbf{Needles:} \begin{itemize}
	\item size 4 needles (24 inch circular)
	\item cable needle (J shape recommended)
	\item tapestry needle
	\item size 2 needle (dpn, tip from interchangeable or circular) for nupps
\end{itemize}

\smallskip
\textbf{Sizing:} As shown, the pattern is repeated 6 times, for a finished diameter of AA and height of BB, to fit a QQ diameter head. To resize, you could work fewer/more repeats of the pattern, and/or adjust the spacing between owls. If you adjust spacing between owls, you will need to modify the increases in chart B and the decreases at the beginning of chart D. I recommend doing chart A over 2/3 the number of stitches used in Chart C. 

\smallskip
\textbf{Swatching:} I recommend working one copy of chart C (plus stockinette border) flat as coaster to get familiar with the owl technique and check sizing. 

\smallskip
\textbf{Notes:}
\begin{itemize}
	\item \textbf{Nupps:} When working the owl eyes, it's important to get the nupp loops loose enough. I recommend holding a small needle (I used size 2 interchangeable tip) double with the right hand needle. Work the nupp loops over both needles, work other stitches in that row only on the normal size 4 needle. 
	\item \textbf{Cables:} This pattern has 4 pair x 5 pair cables (which feel like 8 x 10 cables!) Slipping stitches purlwise and use a J shaped cable needle to get stitches reordered and back on the left hand needle, then work the cable. It will feel very tight. 
\end{itemize}

\smallskip

\textbf{Pattern:}
Cast on 96 pairs of AB color. 

Work Chart A six times around for 15 rows (96 pairs). 

Work Chart B six times around (144 pairs). 

Work Chart C six times around (144 pairs). 

Work Chart D six times around (6 pairs). 

Cinch stitches with yarn ends, knot, and secure ends. Block if desired. 


\doclicenseThis
Contact: vero.boyce@gmail.com

\newpage

\begin{tikzpicture}[scale=\textwidth/25cm,x=-1cm]
  %%% BA pairs %%%
  \foreach \x in {1, 3, ..., 15} {
    \fill[gray] (\x, 0) rectangle (\x + 1, 1);
  }
  
  %%% Column/row labels %%%
  \foreach \x in {1, ..., 16} {
    \node at (\x - 0.5, -0.5) {\x};
  }
  \node at (-0.5, 0.5) {1};
  
  %%% Grid %%%
  \draw[black] (0, 0) rectangle (16, 1);
  \foreach \x in {1, ..., 15} {
    \draw[black] (\x, 0) -- (\x, 1);
  }
\end{tikzpicture}

\begin{tikzpicture}[scale=\textwidth/25cm,x=-1cm]
  %%% Null pairs %%%
  \foreach \x in {2, 3, 10, 11, 14, 15, 22, 23} {
    \fill[black] (\x - 1, 0) rectangle (\x, 2);
  }
  \foreach \x in {3, 10, 15, 22} {
    \fill[black] (\x - 1, 2) rectangle (\x, 4);
  }
  
  %%% Lifted increases %%%
  \foreach \x/\y in {2/3, 2/5, 14/3, 14/5} {
    \draw[black, line width=1] (\x - 0.8, \y - 0.8)
        +(0, 0.6) -- +(0, 0) -- +(0.6, 0.6);
  }
  \foreach \x/\y in {11/3, 11/5, 23/3, 23/5} {
    \draw[black, line width=1] (\x - 0.8, \y - 0.8)
        +(0.6, 0.6) -- +(0.6, 0) -- +(0, 0.6);
  }
  
  %%% Column/row labels %%%
  \foreach \x in {1, ..., 24} {
    \node at (\x - 0.5, -0.5) {\x};
  }
  \foreach \y in {1, ..., 5} {
    \node at (-0.5, \y - 0.5) {\y};
  }
  
  %%% Grid %%%
  \draw[black] (0, 0) rectangle (24, 5);
  \foreach \x in {1, ..., 23} {
    \draw[black] (\x, 0) -- (\x, 5);
  }
  \foreach \y in {1, ..., 4} {
    \draw[black] (0, \y) -- (24, \y);
  }
\end{tikzpicture}

\begin{tikzpicture}[scale=\textwidth/25cm,x=-1cm]
  %%% BA pairs %%%
  \fill[gray] (3, 0) rectangle (8, 3);
  \fill[gray] (16, 0) rectangle (21, 3);
  
  % \fill[gray] (3, 3) rectangle (21, 6);
  \fill[gray] (3, 3) rectangle (21, 7);
  
  \fill[gray] (3, 7) rectangle (8, 15);
  \fill[gray] (16, 7) rectangle (21, 15);
  
  % \fill[gray] (3, 15) rectangle (21, 19);
  \fill[gray] (3, 15) rectangle (21, 20);
  
  \fill[gray] (3, 20) rectangle (10, 23);
  \fill[gray] (11, 20) rectangle (13, 23);
  \fill[gray] (14, 20) rectangle (21, 23);
  
  \fill[gray] (3, 23) rectangle (21, 25);
  \fill[gray] (7, 25) rectangle (17, 26);
  
  \fill[gray] (3, 27) rectangle (8, 28);
  \fill[gray] (16, 27) rectangle (21, 28);
  
  \foreach \x in {9, ..., 16} {
    \foreach \y [var=\sum, evaluate=\sum using \x + \y] in {8, ..., 15} {
      \ifodd\sum
        \fill[gray] (\x, \y) rectangle ++(-1, -1);
      \fi
    }
  }
  
  %%% Nupps %%%
  \node at (11 - 0.5, 22 - 0.5) {N9};
  \node at (14 - 0.5, 22 - 0.5) {N9};
  \draw[black, line width=1] (10, 22)
      +(0.5, 0.2) -- +(0.7, 0.5) -- +(0.5, 0.8) -- +(0.3, 0.5) -- cycle;
  \draw[black, line width=1] (13, 22)
      +(0.5, 0.2) -- +(0.7, 0.5) -- +(0.5, 0.8) -- +(0.3, 0.5) -- cycle;
  
  %%% Column/row labels %%%
  \foreach \x in {1, ..., 24} {
    \node at (\x - 0.5, -0.5) {\x};
  }
  \foreach \y in {1, ..., 28} {
    \node at (-0.5, \y - 0.5) {\y};
  }
  
  %%% Grid %%%
  \draw[black] (0, 0) rectangle (24, 28);
  \foreach \x in {4, ..., 20} {
    % \draw[black] (\x, 0) -- (\x, 6);
    % \draw[black] (\x, 7) -- (\x, 19);
    % \draw[black] (\x, 20) -- (\x, 26);
    % \draw[black] (\x, 27) -- (\x, 28);
    \draw[black] (\x, 0) -- (\x, 28);
  }
  \foreach \x in {1, 2, 3, 21, 22, 23} {
    \draw[black] (\x, 0) -- (\x, 28);
  }
  \foreach \y in {1, ..., 27} {
    \draw[black] (0, \y) -- (24, \y);
  }
  
  %%% Cables %%%
  \foreach \y/\c in {6/gray, 19/gray, 26/white} {
    % \fill[black] (3, \y) rectangle (21, \y + 1);
    
    % \fill[\c] (3, \y) -- ++(4, 0) -- ++(5, 1) -- ++(-4, 0) -- cycle;
    % \draw[black] (4, \y) -- ++(5, 1);
    % \draw[black] (5, \y) -- ++(5, 1);
    % \draw[black] (6, \y) -- ++(5, 1);
    
    \fill[gray, draw=black] (12, \y) -- ++(-4, 1) -- ++(-5, 0) -- ++(4, -1) -- cycle;
    % \draw[black] (7, \y) -- ++(-4, 1);
    \draw[black] (8, \y) -- ++(-4, 1);
    \draw[black] (9, \y) -- ++(-4, 1);
    \draw[black] (10, \y) -- ++(-4, 1);
    \draw[black] (11, \y) -- ++(-4, 1);
    % \draw[black] (12, \y) -- ++(-4, 1);
    
    \fill[gray, draw=black] (12, \y) -- ++(4, 1) -- ++(5, 0) -- ++(-4, -1) -- cycle;
    % \draw[black] (12, \y) -- ++(4, 1);
    \draw[black] (13, \y) -- ++(4, 1);
    \draw[black] (14, \y) -- ++(4, 1);
    \draw[black] (15, \y) -- ++(4, 1);
    \draw[black] (16, \y) -- ++(4, 1);
    % \draw[black] (17, \y) -- ++(4, 1);
  }
\end{tikzpicture}

\begin{tikzpicture}[scale=\textwidth/25cm,x=-1cm]
  %%% BA pairs %%%
  \fill[gray] % Sorry for the unreadability of this. 😕
      (3, 5)
      -- ++(1, 0) -- ++(0, 1) -- ++(1, 0) -- ++(0, 1)
      -- ++(2, 0) -- ++(0, 1) -- ++(1, 0) -- ++(0, 1)
      -- ++(2, 0) -- ++(0, 1) -- ++(1, 0) -- ++(0, 1) -- ++(2, 0) -- ++(0, -1) -- ++(1, 0) -- ++(0, -1) -- ++(2, 0)
      -- ++(0, -1) -- ++(1, 0) -- ++(0, -1) -- ++(2, 0)
      -- ++(0, -1) -- ++(1, 0) -- ++(0, -1) -- ++(1, 0)
      -- ++(0, 2) -- ++(-2, 0)
      -- ++(0, 1) -- ++(-1, 0) -- ++(0, 1) -- ++(-2, 0)
      -- ++(0, 1) -- ++(-1, 0) -- ++(0, 1) -- ++(-1, 0) -- ++(0, 1) -- ++(-1, 0) -- ++(0, 1) -- ++(-2, 0) -- ++(0, -1) -- ++(-1, 0) -- ++(0, -1) -- ++(-1, 0) -- ++(0, -1) -- ++(-1, 0) -- ++(0, -1)
      -- ++(-2, 0) -- ++(0, -1) -- ++(-1, 0) -- ++(0, -1)
      -- ++(-2, 0) -- ++(0, -2)
      
      (6, 11)
      -- ++(1, 0) -- ++(0, 1) -- ++(1, 0) -- ++(0, 1)
      -- ++(2, 0) -- ++(0, 1) -- ++(1, 0) -- ++(0, 1) -- ++(2, 0) -- ++(0, -1) -- ++(1, 0) -- ++(0, -1) -- ++(2, 0)
      -- ++(0, -1) -- ++(1, 0) -- ++(0, -1) -- ++(1, 0)
      -- ++(0, 2) -- ++(-2, 0)
      -- ++(0, 1) -- ++(-1, 0) -- ++(0, 1) -- ++(-1, 0) -- ++(0, 1) -- ++(-1, 0) -- ++(0, 1) -- ++(-2, 0) -- ++(0, -1) -- ++(-1, 0) -- ++(0, -1) -- ++(-1, 0) -- ++(0, -1) -- ++(-1, 0) -- ++(0, -1)
      -- ++(-2, 0) -- ++(0, -2);
  
  %%% Decreases %%%
  \foreach \y in {1, 3, ..., 19} {
    \draw[black, line width=1] (10 + 0.8, \y + 0.2) -- ++(-0.6, 0.6);
  }
  \foreach \y in {1, 3, ..., 19} {
    \draw[black, line width=1] (13 + 0.2, \y + 0.2) -- ++(0.6, 0.6);
  }
  \draw[black, line width=1] (11.8, 21.2) -- ++(-0.6, 0.6);
  \draw[black, line width=1] (12.2, 21.2) -- ++(0.6, 0.6);
  \draw[black, line width=1] (12.2, 23.2) -- ++(0.6, 0.6);
  
  %%% Column/row labels %%%
  \foreach \x in {1, ..., 22} {
    \node at (\x + 0.5, 0.5) {\x};
  }
  \foreach \y [var=\shift, evaluate=\shift using floor((\y - 1)/2)] in {1, ..., 23} {
    \node at (0.5 + \shift, \y + 0.5) {\y};
  }
  
  %%% Grid %%%
  \foreach \x in {1, ..., 11} {
    \draw[black] (\x, 1) -- (\x, \x + \x + 1);
  }
  \draw[black] (12, 1) -- (12, 24);
  \draw[black] (13, 1) -- (13, 24);
  \foreach \x in {14, ..., 23} {
    \draw[black] (\x, 1) -- (\x, 49 - \x - \x);
  }
  
  \draw[black] (1, 1) -- (23, 1);
  \foreach \y [var=\shift, evaluate=\shift using \y/2 - 1] in {2, 4, ..., 23} {
    \draw[black] (1 + \shift, \y) -- (23 - \shift, \y);
    \draw[black] (1 + \shift, \y + 1) -- (23 - \shift, \y + 1);
  }
  \draw[black] (12, 24) -- (13, 24);
\end{tikzpicture}

\end{document}